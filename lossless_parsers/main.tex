\documentclass[compress,12pt,xcolor={dvipsnames}]{beamer}

\usepackage{pifont}
\usepackage{enumitem}
\usepackage[T1]{fontenc}
\usepackage[utf8]{inputenc}
\usepackage{listings}
\usepackage{amssymb}
\usepackage{color}

\newlist{checklist}{itemize}{2}
\setlist[checklist]{label=$\square$}
\newcommand{\cmark}{\ding{51}}%
\newcommand{\xmark}{\ding{55}}%
\newcommand{\done}{\rlap{$\square$}{\raisebox{2pt}{\large\hspace{1pt}\cmark}}%
\hspace{-2.5pt}}
\newcommand{\wontfix}{\rlap{$\square$}{\large\hspace{1pt}\xmark}}

% \addtobeamertemplate{note page}{}{\thispdfpagelabel{notes:\insertframenumber}}
% \setbeameroption{show notes on second screen}

\graphicspath{ {./assets/} }

\usetheme{Arguelles}

\title{Lossless/Error-tolerant parsers}
\subtitle{Why and how}
\event{}
\date{}
\author{Marcin Wojnarowski}
\institute{PL seminars}
% TODO: Does not work
% \email{name@domain.com}
% \homepage{www.mywebsite.com}
% \github{username}

\definecolor{keywordcolor}{rgb}{0.7, 0.1, 0.1}   % red
\definecolor{tacticcolor}{rgb}{0.0, 0.1, 0.6}    % blue
\definecolor{commentcolor}{rgb}{0.4, 0.4, 0.4}   % grey
\definecolor{symbolcolor}{rgb}{0.0, 0.1, 0.6}    % blue
\definecolor{sortcolor}{rgb}{0.1, 0.5, 0.1}      % green
\definecolor{attributecolor}{rgb}{0.7, 0.1, 0.1} % red

\def\lstlanguagefiles{lstlean.tex}
\lstdefinelanguage{Oop}{
  keywords={func, class, extends, var, throw, if, self, return, override, constructor},
  ndkeywords={true, false},
  sensitive=false,
  comment=[l]{\#},
}

\lstdefinelanguage{typescript}{
  keywords={type, function, if, else, typeof, satisfies},
  ndkeywords={true, false},
  sensitive=false,
  comment=[l]{\#},
}

\lstdefinelanguage{Cpp}{
  keywords={\#include},
  ndkeywords={int},
  sensitive=false,
  comment=[l]{//},
}

\lstdefinelanguage{TypeState}{
  keywords={func, class, typestate, transition_to, when, with},
  emph={int, string},
  sensitive=false,
  comment=[l]{\#},
}

\lstdefinelanguage{Koka}{
  keywords={effect, ctl, fun, val, var, return, if, handler, then},
  emph={bool, False, True, println},
  sensitive=false,
  comment=[l]{\#},
}

\lstdefinelanguage{Dart}{
  keywords={class, if, else},
  emph={int, string, void, print},
  sensitive=false,
  comment=[l]{//},
}

\lstdefinelanguage{Kotlin}{
  comment=[l]{//},
  emph={filter, first, firstOrNull, forEach, lazy, map, mapNotNull, println},
  keywords={!in, !is, abstract, actual, annotation, as, as?, break, by, catch, class, companion, const, constructor, continue, crossinline, data, delegate, do, dynamic, else, enum, expect, external, false, field, file, final, finally, for, fun, get, if, import, in, infix, init, inline, inner, interface, internal, is, lateinit, noinline, null, object, open, operator, out, override, package, param, private, property, protected, public, receiveris, reified, return, return@, sealed, set, setparam, super, suspend, tailrec, this, throw, true, try, typealias, typeof, val, var, vararg, when, where, while},
  morecomment=[s]{/*}{*/},
  morestring=[b]",
  morestring=[s]{"""*}{*"""},
  ndkeywords={@Deprecated, @JvmField, @JvmName, @JvmOverloads, @JvmStatic, @JvmSynthetic, Array, Byte, Double, Float, Int, Integer, Iterable, Long, Runnable, Short, String, Any, Unit, Nothing},
}

\lstdefinelanguage{Rust}{%
sensitive%
, morecomment=[l]{//}%
, morecomment=[s]{/*}{*/}%
, moredelim=[s][{\itshape\color[rgb]{0,0,0.75}}]{\#[}{]}%
, morestring=[b]{"}%
, alsodigit={}%
, alsoother={}%
, alsoletter={!}%
%
%
% [1] reserve keywords
% [2] traits
% [3] primitive types
% [4] type and value constructors
% [5] identifier
%
, morekeywords={break, continue, else, for, if, in, loop, match, return, while}  % control flow keywords
, morekeywords={as, const, let, move, mut, ref, static}  % in the context of variables
, morekeywords={dyn, enum, fn, impl, Self, self, struct, trait, type, union, use, where}  % in the context of declarations
, morekeywords={crate, extern, mod, pub, super}  % in the context of modularisation
, morekeywords={unsafe}  % markers
, morekeywords={abstract, alignof, become, box, do, final, macro, offsetof, override, priv, proc, pure, sizeof, typeof, unsized, virtual, yield}  % reserved identifiers
%
% grep 'pub trait [A-Za-z][A-Za-z0-9]*' -r . | sed 's/^.*pub trait \([A-Za-z][A-Za-z0-9]*\).*/\1/g' | sort -u | tr '\n' ',' | sed 's/^\(.*\),$/{\1}\n/g' | sed 's/,/, /g'
, morekeywords=[2]{Add, AddAssign, Any, AsciiExt, AsInner, AsInnerMut, AsMut, AsRawFd, AsRawHandle, AsRawSocket, AsRef, Binary, BitAnd, BitAndAssign, Bitor, BitOr, BitOrAssign, BitXor, BitXorAssign, Borrow, BorrowMut, Boxed, BoxPlace, BufRead, BuildHasher, CastInto, CharExt, Clone, CoerceUnsized, CommandExt, Copy, Debug, DecodableFloat, Default, Deref, DerefMut, DirBuilderExt, DirEntryExt, Display, Div, DivAssign, DoubleEndedIterator, DoubleEndedSearcher, Drop, EnvKey, Eq, Error, ExactSizeIterator, ExitStatusExt, Extend, FileExt, FileTypeExt, Float, Fn, FnBox, FnMut, FnOnce, Freeze, From, FromInner, FromIterator, FromRawFd, FromRawHandle, FromRawSocket, FromStr, FullOps, FusedIterator, Generator, Hash, Hasher, Index, IndexMut, InPlace, Int, Into, IntoCow, IntoInner, IntoIterator, IntoRawFd, IntoRawHandle, IntoRawSocket, IsMinusOne, IsZero, Iterator, JoinHandleExt, LargeInt, LowerExp, LowerHex, MetadataExt, Mul, MulAssign, Neg, Not, Octal, OpenOptionsExt, Ord, OsStrExt, OsStringExt, Packet, PartialEq, PartialOrd, Pattern, PermissionsExt, Place, Placer, Pointer, Product, Put, RangeArgument, RawFloat, Read, Rem, RemAssign, Seek, Shl, ShlAssign, Shr, ShrAssign, Sized, SliceConcatExt, SliceExt, SliceIndex, Stats, Step, StrExt, Sub, SubAssign, Sum, Sync, TDynBenchFn, Terminal, Termination, ToOwned, ToSocketAddrs, ToString, Try, TryFrom, TryInto, UnicodeStr, Unsize, UpperExp, UpperHex, WideInt, Write}
, morekeywords=[2]{Send}  % additional traits
%
, morekeywords=[3]{bool, char, f32, f64, i8, i16, i32, i64, isize, str, u8, u16, u32, u64, unit, usize, i128, u128}  % primitive types
%
, morekeywords=[4]{Err, false, None, Ok, Some, true}  % prelude value constructors
% grep 'pub \(type\|struct\|enum\) [A-Za-z][A-Za-z0-9]*' -r . | sed 's/^.*pub \(type\|struct\|enum\) \([A-Za-z][A-Za-z0-9]*\).*/\2/g' | sort -u | tr '\n' ',' | sed 's/^\(.*\),$/{\1}\n/g' | sed 's/,/, /g'    
, morekeywords=[3]{AccessError, Adddf3, AddI128, AddoI128, AddoU128, ADDRESS, ADDRESS64, addrinfo, ADDRINFOA, AddrParseError, Addsf3, AddU128, advice, aiocb, Alignment, AllocErr, AnonPipe, Answer, Arc, Args, ArgsInnerDebug, ArgsOs, Argument, Arguments, ArgumentV1, Ashldi3, Ashlti3, Ashrdi3, Ashrti3, AssertParamIsClone, AssertParamIsCopy, AssertParamIsEq, AssertUnwindSafe, AtomicBool, AtomicPtr, Attr, auxtype, auxv, BackPlace, BacktraceContext, Barrier, BarrierWaitResult, Bencher, BenchMode, BenchSamples, BinaryHeap, BinaryHeapPlace, blkcnt, blkcnt64, blksize, BOOL, boolean, BOOLEAN, BoolTrie, BorrowError, BorrowMutError, Bound, Box, bpf, BTreeMap, BTreeSet, Bucket, BucketState, Buf, BufReader, BufWriter, Builder, BuildHasherDefault, BY, BYTE, Bytes, CannotReallocInPlace, cc, Cell, Chain, CHAR, CharIndices, CharPredicateSearcher, Chars, CharSearcher, CharsError, CharSliceSearcher, CharTryFromError, Child, ChildPipes, ChildStderr, ChildStdin, ChildStdio, ChildStdout, Chunks, ChunksMut, ciovec, clock, clockid, Cloned, cmsgcred, cmsghdr, CodePoint, Color, ColorConfig, Command, CommandEnv, Component, Components, CONDITION, condvar, Condvar, CONSOLE, CONTEXT, Count, Cow, cpu, CRITICAL, CStr, CString, CStringArray, Cursor, Cycle, CycleIter, daddr, DebugList, DebugMap, DebugSet, DebugStruct, DebugTuple, Decimal, Decoded, DecodeUtf16, DecodeUtf16Error, DecodeUtf8, DefaultEnvKey, DefaultHasher, dev, device, Difference, Digit32, DIR, DirBuilder, dircookie, dirent, dirent64, DirEntry, Discriminant, DISPATCHER, Display, Divdf3, Divdi3, Divmoddi4, Divmodsi4, Divsf3, Divsi3, Divti3, dl, Dl, Dlmalloc, Dns, DnsAnswer, DnsQuery, dqblk, Drain, DrainFilter, Dtor, Duration, DwarfReader, DWORD, DWORDLONG, DynamicLibrary, Edge, EHAction, EHContext, Elf32, Elf64, Empty, EmptyBucket, EncodeUtf16, EncodeWide, Entry, EntryPlace, Enumerate, Env, epoll, errno, Error, ErrorKind, EscapeDebug, EscapeDefault, EscapeUnicode, event, Event, eventrwflags, eventtype, ExactChunks, ExactChunksMut, EXCEPTION, Excess, ExchangeHeapSingleton, exit, exitcode, ExitStatus, Failure, fd, fdflags, fdsflags, fdstat, ff, fflags, File, FILE, FileAttr, filedelta, FileDesc, FilePermissions, filesize, filestat, FILETIME, filetype, FileType, Filter, FilterMap, Fixdfdi, Fixdfsi, Fixdfti, Fixsfdi, Fixsfsi, Fixsfti, Fixunsdfdi, Fixunsdfsi, Fixunsdfti, Fixunssfdi, Fixunssfsi, Fixunssfti, Flag, FlatMap, Floatdidf, FLOATING, Floatsidf, Floatsisf, Floattidf, Floattisf, Floatundidf, Floatunsidf, Floatunsisf, Floatuntidf, Floatuntisf, flock, ForceResult, FormatSpec, Formatted, Formatter, Fp, FpCategory, fpos, fpos64, fpreg, fpregset, FPUControlWord, Frame, FromBytesWithNulError, FromUtf16Error, FromUtf8Error, FrontPlace, fsblkcnt, fsfilcnt, fsflags, fsid, fstore, fsword, FullBucket, FullBucketMut, FullDecoded, Fuse, GapThenFull, GeneratorState, gid, glob, glob64, GlobalDlmalloc, greg, group, GROUP, Guard, GUID, Handle, HANDLE, Handler, HashMap, HashSet, Heap, HINSTANCE, HMODULE, hostent, HRESULT, id, idtype, if, ifaddrs, IMAGEHLP, Immut, in, in6, Incoming, Infallible, Initializer, ino, ino64, inode, input, InsertResult, Inspect, Instant, int16, int32, int64, int8, integer, IntermediateBox, Internal, Intersection, intmax, IntoInnerError, IntoIter, IntoStringError, intptr, InvalidSequence, iovec, ip, IpAddr, ipc, Ipv4Addr, ipv6, Ipv6Addr, Ipv6MulticastScope, Iter, IterMut, itimerspec, itimerval, jail, JoinHandle, JoinPathsError, KDHELP64, kevent, kevent64, key, Key, Keys, KV, l4, LARGE, lastlog, launchpad, Layout, Lazy, lconv, Leaf, LeafOrInternal, Lines, LinesAny, LineWriter, linger, linkcount, LinkedList, load, locale, LocalKey, LocalKeyState, Location, lock, LockResult, loff, LONG, lookup, lookupflags, LookupHost, LPBOOL, LPBY, LPBYTE, LPCSTR, LPCVOID, LPCWSTR, LPDWORD, LPFILETIME, LPHANDLE, LPOVERLAPPED, LPPROCESS, LPPROGRESS, LPSECURITY, LPSTARTUPINFO, LPSTR, LPVOID, LPWCH, LPWIN32, LPWSADATA, LPWSAPROTOCOL, LPWSTR, Lshrdi3, Lshrti3, lwpid, M128A, mach, major, Map, mcontext, Metadata, Metric, MetricMap, mflags, minor, mmsghdr, Moddi3, mode, Modsi3, Modti3, MonitorMsg, MOUNT, mprot, mq, mqd, msflags, msghdr, msginfo, msglen, msgqnum, msqid, Muldf3, Mulodi4, Mulosi4, Muloti4, Mulsf3, Multi3, Mut, Mutex, MutexGuard, MyCollection, n16, NamePadding, NativeLibBoilerplate, nfds, nl, nlink, NodeRef, NoneError, NonNull, NonZero, nthreads, NulError, OccupiedEntry, off, off64, oflags, Once, OnceState, OpenOptions, Option, Options, OptRes, Ordering, OsStr, OsString, Output, OVERLAPPED, Owned, Packet, PanicInfo, Param, ParseBoolError, ParseCharError, ParseError, ParseFloatError, ParseIntError, ParseResult, Part, passwd, Path, PathBuf, PCONDITION, PCONSOLE, Peekable, PeekMut, Permissions, PhantomData, pid, Pipes, PlaceBack, PlaceFront, PLARGE, PoisonError, pollfd, PopResult, port, Position, Powidf2, Powisf2, Prefix, PrefixComponent, PrintFormat, proc, Process, PROCESS, processentry, protoent, PSRWLOCK, pthread, ptr, ptrdiff, PVECTORED, Queue, radvisory, RandomState, Range, RangeFrom, RangeFull, RangeInclusive, RangeMut, RangeTo, RangeToInclusive, RawBucket, RawFd, RawHandle, RawPthread, RawSocket, RawTable, RawVec, Rc, ReadDir, Receiver, recv, RecvError, RecvTimeoutError, ReentrantMutex, ReentrantMutexGuard, Ref, RefCell, RefMut, REPARSE, Repeat, Result, Rev, Reverse, riflags, rights, rlim, rlim64, rlimit, rlimit64, roflags, Root, RSplit, RSplitMut, RSplitN, RSplitNMut, RUNTIME, rusage, RwLock, RWLock, RwLockReadGuard, RwLockWriteGuard, sa, SafeHash, Scan, sched, scope, sdflags, SearchResult, SearchStep, SECURITY, SeekFrom, segment, Select, SelectionResult, sem, sembuf, send, Sender, SendError, servent, sf, Shared, shmatt, shmid, ShortReader, ShouldPanic, Shutdown, siflags, sigaction, SigAction, sigevent, sighandler, siginfo, Sign, signal, signalfd, SignalToken, sigset, sigval, Sink, SipHasher, SipHasher13, SipHasher24, size, SIZE, Skip, SkipWhile, Slice, SmallBoolTrie, sockaddr, SOCKADDR, sockcred, Socket, SOCKET, SocketAddr, SocketAddrV4, SocketAddrV6, socklen, speed, Splice, Split, SplitMut, SplitN, SplitNMut, SplitPaths, SplitWhitespace, spwd, SRWLOCK, ssize, stack, STACKFRAME64, StartResult, STARTUPINFO, stat, Stat, stat64, statfs, statfs64, StaticKey, statvfs, StatVfs, statvfs64, Stderr, StderrLock, StderrTerminal, Stdin, StdinLock, Stdio, StdioPipes, Stdout, StdoutLock, StdoutTerminal, StepBy, String, StripPrefixError, StrSearcher, subclockflags, Subdf3, SubI128, SuboI128, SuboU128, subrwflags, subscription, Subsf3, SubU128, Summary, suseconds, SYMBOL, SYMBOLIC, SymmetricDifference, SyncSender, sysinfo, System, SystemTime, SystemTimeError, Take, TakeWhile, tcb, tcflag, TcpListener, TcpStream, TempDir, TermInfo, TerminfoTerminal, termios, termios2, TestDesc, TestDescAndFn, TestEvent, TestFn, TestName, TestOpts, TestResult, Thread, threadattr, threadentry, ThreadId, tid, time, time64, timespec, TimeSpec, timestamp, timeval, timeval32, timezone, tm, tms, ToLowercase, ToUppercase, TraitObject, TryFromIntError, TryFromSliceError, TryIter, TryLockError, TryLockResult, TryRecvError, TrySendError, TypeId, U64x2, ucontext, ucred, Udivdi3, Udivmoddi4, Udivmodsi4, Udivmodti4, Udivsi3, Udivti3, UdpSocket, uid, UINT, uint16, uint32, uint64, uint8, uintmax, uintptr, ulflags, ULONG, ULONGLONG, Umoddi3, Umodsi3, Umodti3, UnicodeVersion, Union, Unique, UnixDatagram, UnixListener, UnixStream, Unpacked, UnsafeCell, UNWIND, UpgradeResult, useconds, user, userdata, USHORT, Utf16Encoder, Utf8Error, Utf8Lossy, Utf8LossyChunk, Utf8LossyChunksIter, utimbuf, utmp, utmpx, utsname, uuid, VacantEntry, Values, ValuesMut, VarError, Variables, Vars, VarsOs, Vec, VecDeque, vm, Void, WaitTimeoutResult, WaitToken, wchar, WCHAR, Weak, whence, WIN32, WinConsole, Windows, WindowsEnvKey, winsize, WORD, Wrapping, wrlen, WSADATA, WSAPROTOCOL, WSAPROTOCOLCHAIN, Wtf8, Wtf8Buf, Wtf8CodePoints, xsw, xucred, Zip, zx}
%
, morekeywords=[5]{assert!, assert_eq!, assert_ne!, cfg!, column!, compile_error!, concat!, concat_idents!, debug_assert!, debug_assert_eq!, debug_assert_ne!, env!, eprint!, eprintln!, file!, format!, format_args!, include!, include_bytes!, include_str!, line!, module_path!, option_env!, panic!, print!, println!, select!, stringify!, thread_local!, try!, unimplemented!, unreachable!, vec!, write!, writeln!}  % prelude macros
}%

\lstdefinelanguage{CSharp}{
  morecomment=[l]{//}, %use comment-line-style!
  morecomment=[l]{\#}, %use comment-line-style!
  morecomment=[s]{/*}{*/}, %for multiline comments
  morekeywords={ abstract, event, new, struct,
      as, explicit, null, switch,
      base, extern, object, this,
      bool, false, operator, throw,
      break, finally, out, true,
      byte, fixed, override, try,
      case, float, params, typeof,
      catch, for, private, uint,
      char, foreach, protected, ulong,
      checked, goto, public, unchecked,
      class, if, readonly, unsafe,
      const, implicit, ref, ushort,
      continue, in, return, using,
      decimal, int, sbyte, virtual,
      default, interface, sealed, volatile,
      delegate, internal, short, void,
      do, is, sizeof, while,
      double, lock, stackalloc,
      else, long, static,
      enum, namespace, string, var},
  morestring=[b]",
}

\lstset{
  breaklines=true,
  numbers=left,
  numberstyle=\scriptsize\color{black},
  escapechar=@,
  commentstyle={\color{gray}\ttfamily},
  emphstyle={\color{OrangeRed}},
  identifierstyle=\color{black}\ttfamily,
  keywordstyle={\color{NavyBlue}\bfseries},
  sensitive=true,
  stringstyle={\color{ForestGreen}\ttfamily},
  basicstyle=\linespread{0.5}\tt\footnotesize
}


% make urls highlighted
\let\oldhref\href
\renewcommand{\href}[2]{\oldhref{#1}{\textcolor{magenta}{#2}}}

\begin{document}

\frame[plain]{\titlepage
	\note[item]{All of this was a result of my analysis-paralysis when creating a parser for my language. It had a requirement that not many languages have: being usable (i.e. good UX) by human beings}
}


\begin{frame}
	\begin{block}{Lossless}
		Parser does not discard any information, AST (well, CST) is full-fidelity i.e. we can reconstruct the exact source code.\\\textbf{Not lossless}: discarding whitespace.\note[item]{We wont talk about lexers, distinction between a lexer and parser is unimportant here}
	\end{block}
	\begin{block}{Error-tolerant}
		Parser doesn't bail on errors, AST represents an approximated tree.\\\textbf{Not error-tolerant}: stopping parsing on first error.
	\end{block}
\end{frame}


\begin{frame}[standout]
	\centering\large
	Why
\end{frame}

\section{Why}
\begin{frame}
	\frametitle{Why lossless?}

	\begin{enumerate}
		\item Sometimes "useless" source code is needed by tools
		\item Code formatters need whitespace information
		\item Code transformers (for example quick fixes in IDEs)
	\end{enumerate}

	From now, information which is conventionally discarded by a parser but we want to keep will be called \textbf{trivia} (most commonly whitespace and comments).
\end{frame}

\begin{frame}
	\frametitle{Why error-tolerant?}

	\begin{enumerate}
		\item Stopping the whole compiler pipeline on the first syntax error is bad UX
		\item LSPs would be unusable otherwise (for example autocomplete inherently operates on incomplete source code)
		\item We can raise diagnostics from future compiler-frontend stages
	\end{enumerate}

	One could wish error-tolerance for other compiler-frontend stages. For example in Rust the borrow checker does not run if the type checker doesn't accept the code.\note[item]{I often fight with the type checker only to end up being completely rejected by the borrow checker}
\end{frame}

\begin{frame}[standout]
	\centering\large
	(Demos)
\end{frame}

\begin{frame}[fragile]
	\frametitle{Which issues are detected in different compilers?}
	\begin{lstlisting}[gobble=8,language=typescript,basicstyle=\tt\footnotesize]
        function func() { return 1; }

        const t: string = func(12 + )

        const a: number = "asd"
    \end{lstlisting}

	\begin{center}
		\begin{tabular}{||c| c c c c||}
			\hline
			Lang       & No expr    & \#Args     & \texttt{func} type & Expr type  \\
			\hline\hline
			Dart       & \checkmark & \checkmark & \checkmark         & \checkmark \\
			\hline
			OCaml      & \checkmark & \xmark     & \xmark             & \xmark     \\
			\hline
			Scala      & \checkmark & \xmark     & \xmark             & \checkmark \\
			\hline
			Swift      & \checkmark & \xmark     & \checkmark         & \checkmark \\
			\hline
			TypeScript & \checkmark & \checkmark & \checkmark         & \checkmark \\
			\hline
			Rust       & \checkmark & \xmark     & \xmark             & \checkmark \\
			\hline
		\end{tabular}
	\end{center}\note[item]{Possibly overloading of functions makes some compilers not able to produce some more precise errors}
\end{frame}

\begin{frame}
	\frametitle{What about their language servers?}

	\begin{itemize}
		\item \textbf{Dart} -- compiler frontend is a library, used by the LSP
		\item \textbf{OCaml} -- \href{https://arxiv.org/pdf/1807.06702}{Merlin} is the language server, extends the existing lexer/parser to provide error recovery and incrementality\note[item]{https://arxiv.org/pdf/1807.06702}		\item \textbf{Scala} -- Metals reimplements the syntactic analyzer\note[item]{It might be doing a flycheck like rust, no type feedback is given until a file is saved}
		\item \textbf{Swift} -- Swift Syntax/IndexStoreDB aim to be the compiler syntactic library, used by LSP but not compiler\note[item]{Compiler is in C++, but libraries in pure Swift}
		\item \textbf{TypeScript} -- Compiler is usable as a library, used by LSP
		\item \textbf{Rust} -- LSP reimplements all syntactic features, type checking is dispatched to the compiler binary
	\end{itemize}
	\note[item]{Other notable example: Roslyn. A big effort to make C\# and VB compiler a library}
\end{frame}

\begin{frame}
	\frametitle{What about their code formatters?}

	\begin{itemize}
		\item \textbf{Dart} -- Uses compiler libraries
		\item \textbf{OCaml} -- \texttt{ocamlformat} forks the parser and augments the AST with comments
		\item \textbf{Scala} -- \texttt{scalafmt} uses scalameta like Metals
		\item \textbf{Swift} -- \texttt{swift-format} uses Swift Syntax
		\item \textbf{TypeScript} -- Many formatters available, AFAIK all reimplement the parser
		\item \textbf{Rust} -- 	\texttt{rustfmt} uses the compiler parser, but due to the AST being lossy authors admit that formatting is only \href{https://github.com/rust-lang/rustfmt/blob/master/Contributing.md\#a-quick-tour-of-rustfmt}{best-effort}
	\end{itemize}

	\note[item]{Other notable example: Roslyn. A big effort to make C\# and VB compiler a library}
\end{frame}


\section{How}

\begin{frame}[standout]
	\centering\large
	How
\end{frame}

\subsection{ASTs}

\begin{frame}
	\frametitle{Importance of ASTs}
	\begin{itemize}
		\item A good AST is needed to be able to support the lossless/error-resilient parser
		\item ASTs need to be full-fidelity and be able to represent possibly erroneous parse nodes
		\item AST nodes need to hold trivia
		\item Half of the art is in the ASTs
	\end{itemize}
\end{frame}


\begin{frame}[fragile]
	\frametitle{Approach 1: hand-crafted AST}
	\begin{itemize}
		\item Usually too subtle and tedious to write by hand, use generators (of AST not parsers!)
		\item Approach taken by \href{https://github.com/dotnet/roslyn/blob/Visual-Studio-2019-Version-16.11/src/Compilers/CSharp/Portable/Generated/CSharpSyntaxGenerator/CSharpSyntaxGenerator.SourceGenerator/Syntax.xml.Internal.Generated.cs}{Roslyn}
		      \begin{lstlisting}[language={[Sharp]C}, gobble=12, numbers=off]
                  RetType MyFunc(ArgType arg) {body}
              \end{lstlisting}
		\item Approach taken by \href{https://github.com/apple/swift-syntax/blob/caab6bb82e0e81395551ec1032984f4b325f7a35/Sources/SwiftSyntax/generated/syntaxNodes/SyntaxNodesEF.swift#L3368}{SwiftSyntax}
		      \begin{lstlisting}[language=Swift, gobble=18, numbers=off]
                  func myFunc(arg: ArgType) -> RetType {body}
              \end{lstlisting}
	\end{itemize}
\end{frame}


\begin{frame}[fragile]
	\frametitle{Approach 2: homogenous AST}
	\begin{itemize}
		\item Using an untyped syntax tree as base
		\item Providing a typed view into the syntax tree
		\item Less obvious how to safely construct AST nodes
		\item Can be reused for various languages
		\item Approach taken by \href{https://github.com/rust-lang/rust-analyzer/blob/5bf2f85c8054d80424899fa581db1b192230efb5/crates/syntax/src/ast/generated/nodes.rs#L357}{Rust analyzer}
		      \begin{lstlisting}[language=Rust, gobble=18, numbers=off]
                  fn my_func(arg: ArgType) -> RetType {body}
              \end{lstlisting}
	\end{itemize}
\end{frame}


% \begin{frame}
% 	\frametitle{Editing the syntax tree?}
%
% \end{frame}

% TODO: green/red tree


\section{Conclusions}

\begin{frame}
	\frametitle{Conclusions}
	\begin{itemize}
		\item Properties of a parser often leak into the design of the AST
		\item A lossless/error-resilient parser makes the compiler frontend usable as a library
		\item Error recovery is best when specialized to the semantics of the language
	\end{itemize}
\end{frame}

\End
\begin{frame}
	\frametitle{Incremental parsers?}

	Some parser are written to be incremental, i.e. we can reparse only some bounded piece of code of the changed fragment instead of the whole file.

	This complicates the parser and the benefits are unclear\note[item]{A good example is the contextual keyword "await", when adding async somewhere higher up the tree, "await" changes its meaning}. Usually parsing is fast enough to reparse the whole file from scratch.
\end{frame}
% https://github.com/leanprover/lean4/blob/26a1b934c226c6aa923248804ba92c23f56c5115/src/Lean/Parser/Basic.lean#L10
% https://github.com/leanprover/lean4/blob/26a1b934c226c6aa923248804ba92c23f56c5115/src/Init/Prelude.lean#L3674
\end{document}
